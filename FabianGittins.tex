\let\nofiles\relax % This is because res says to not emit aux files,
                   % but lastpage needs aux files.

\documentclass[%
    margin,
    line,
    11pt,
]{res}

\usepackage[style=iso]{datetime2}
\usepackage{etaremune}
\usepackage{fancyhdr}
\usepackage[T1]{fontenc}
\usepackage{fourier-orns}
\usepackage[colorlinks=true]{hyperref}
\usepackage{lastpage}
\usepackage{microtype}
\usepackage{newtxtext,newtxmath}

\newsectionwidth{1.5in}

\oddsidemargin -0.5in
\voffset -25pt
\headsep 25pt
\textwidth 6.0in

% Headings
\pagestyle{fancy}
\lhead{Fabian Gittins --- Curriculum Vitae (as of \today)}
\chead{}
\rhead{\thepage\ of \pageref*{LastPage}}
\lfoot{}
\cfoot{}
\rfoot{}

% Give hyperref some metadata
\hypersetup{%
    pdftitle={Fabian Gittins — Curriculum Vitae (CV)},
    pdfauthor={Fabian Gittins},
    pdfsubject={Fabian Gittins's academic curriculum vitae (CV)},
    pdfkeywords={physics, gravity, general relativity, neutron stars,
    gravitational waves, asteroseismology}
}

% I want small caps for the section style
\def\sectionfont{\sc}

% Generate a PDF TOC
\let\oldsection\section
\def\section#1{%
    \oldsection{#1}%
    \phantomsection%
    \addcontentsline{toc}{section}{#1}%
}

% The above seems to screw up the spacing for sections that start with lists...
\newcommand{\secstartswithlist}{\leavevmode \vspace{-\baselineskip}}

\begin{document}

\newcommand{\myname}{Fabian Gittins}
\newlength{\mynamewidth}
\settowidth{\mynamewidth}{\namefont\myname}

\name{\hspace*{0.5\textwidth}\hspace{-0.5\mynamewidth} \myname \vspace*{0.1in}}
% On the first page, have no header.
\thispagestyle{empty}

\begin{resume}

\section{Contact Information}

%\vspace{0.05in}
Institute for Gravitational and Subatomic Physics
\hfill \href{mailto:f.w.r.gittins@uu.nl}{f.w.r.gittins@uu.nl} \\
%
Princetonplein 1, Utrecht University
\hfill \href{https://fgittins.github.io}{fgittins.github.io} \\
%
3584 CC Utrecht, The Netherlands
\hfill \href{tel:+31 6 57 918 906}{+31 6 57 918 906}

%%%%%%%%%%%%%%%%%%%%%%%%%%%%%%%%%%%%%%%%%%%%

% Local Variables:
% mode: latex
% TeX-master: "FabianGittins.tex"
% End:


\section{Citizenship}

United Kingdom

\section{Education}

\textbf{PhD, Mathematics,} University of Southampton, UK
\hfill \textbf{Sep. 2021} \\
\vspace*{-0.1in}
\begin{itemize}
    \item[] Advisor: Prof Nils Andersson
    %
    \item[] Thesis title:
    \href{https://eprints.soton.ac.uk/452900/}%
    {\textit{Gravitational waves from deformed neutron stars: mountains and tides}}
\end{itemize}
%
\textbf{MSci, Physics,} University of Birmingham, UK
\hfill \textbf{Jul. 2017} \\
\vspace*{-0.1in}
\begin{itemize}
    \item[] Grade: First class honours
    %
    \item[] Undergraduate Master's degree with focus on theoretical physics
\end{itemize}

\section{Research Experience}

\textbf{Marie Sk{\l}odowska-Curie Postdoctoral Fellow,} Utrecht University, NL
\hfill \textbf{Oct. 2024--Present} \\
\\
\textbf{Research Fellow,} University of Southampton, UK
\hfill \textbf{Oct. 2021--Sep. 2024} \\
\\
\textbf{PhD Researcher,} University of Southampton, UK
\hfill \textbf{Sep. 2017--Sep. 2024} \\
\\
\textbf{Undergraduate Researcher,} University of Birmingham, UK
\hfill \textbf{Sep. 2016--Mar. 2017} \\
\\
\textbf{Undergraduate Researcher,} University of Birmingham, UK
\hfill \textbf{Jan.--May 2016} \\
\\
\textbf{Undergraduate Research Intern,} University of Birmingham, UK
\hfill \textbf{Jun.--Sep. 2015} \\

% Added to improve page breaks
\vspace{-1em}

\section{Research Interests}

Relativistic astrophysics, gravitational-wave astronomy and the extreme physics
of neutron stars.
One major theme is building predictive, physically faithful neutron-star models
with realistic microphysics and dynamics.
A second major theme is using gravitational waves to extract this physics and
constrain dense nuclear matter.
Currently, advancing gravitational-wave asteroseismology to probe dense-matter
physics through the modelling and detection of neutron-star oscillation modes.

\section{Grants and Awards}

\textbf{Marie Sk{\l}odowska-Curie Postdoctoral Fellowship,} European Union
\hfill \textbf{Oct. 2024--Sep. 2026}\vspace*{0.05in}
\begin{itemize}
    \item[] Project lead of
    \href{https://doi.org/10.3030/101151301}{\textit{DynTideEOS}}; \eurologo 203,464
\end{itemize}
%
\textbf{Gravitational Physics Thesis Prize,} Institute of Physics, UK
\hfill \textbf{2021} \\
\\
\textbf{Best Publication in Gravitational Physics,}
University of Southampton, UK
\hfill \textbf{2021} \\
\\
\textbf{Physics Scholarship,} University of Birmingham, UK
\hfill \textbf{2013} \\

\newpage

%\newcommand{\playsymbol}{\framebox[1.3\width]{$\blacktriangleright$}}
\newcommand{\playsymbol}{$\blacktriangleright$}

\section{Invited Talks}

\secstartswithlist{}%
\begin{etaremune}
    \item \textit{High Energy Particle Physics and Cosmology Theory Seminar},
    \hfill \textbf{30 Sep. 2025} \\
    The Johns Hopkins University, Baltimore, USA
    %
    \item \textit{Institute for Nuclear Theory Program 25-2b}
    \hfill \textbf{17 Sep. 2025} \\
    University of Washington, Seattle, USA
    %
    \item \textit{Gravitational Wave Meeting}, \hfill \textbf{18 Jun. 2025} \\
    National Institute for Subatomic Physics, NL (online)
    %
    \item \textit{Astrophysics Seminar}, \hfill \textbf{30 May 2024} \\
    Mullard Space Science Laboratory, University College London, UK
    %
    \item \textit{Gravitational Wave Group}, \hfill \textbf{14 Dec. 2023} \\
    Institute of Cosmology and Gravitation, University of Portsmouth, UK
    %
    \item \textit{SPINS-UK Seminar} (online) \hfill \textbf{7 Jun. 2023}
    %
    \item
    \textit{Symposium on Gravitational Wave Physics and Astronomy: Genesis},
    \hfill \textbf{28 Apr. 2022} \\
    Kyoto University, JP (online)
    %
    \item \textit{22nd BritGrav Conference}, University of Glasgow, UK (online)
    \hfill \textbf{5 Apr. 2022}
    %
    \item \textit{Colloquium}, Albert Einstein Institute, Hannover, DE (online)
    \hfill \textbf{6 Oct. 2020}
    %
    \item \textit{LIGO-Virgo Collaboration Continuous Waves Working Group}
    (online) \hfill \textbf{5 Dec. 2018}
\end{etaremune}

\section{Contributed Talks (Selected)}

24 contributed talks at 22 separate conferences and meetings, including

\begin{etaremune}
    \item Joint \textit{24th International Conference on General Relativity and
    Gravitation} \hfill \textbf{17 Jul. 2025} \\
    and \textit{16th Edoardo Amaldi Conference on Gravitational Waves}, Glasgow, UK
    %
    \item \textit{XV Einstein Telescope Symposium}, Bologna, IT
    \hfill \textbf{27 May 2025}
    %
    \item \textit{Institute for Nuclear Theory Workshop 24-89w},
    \hfill \textbf{5 Sep. 2024} \\
    University of Washington, Seattle, USA
    %
    \item \textit{XIV Einstein Telescope Symposium}, Maastricht, NL
    \hfill \textbf{6--7 May 2025}
    %
    \item \textit{SPINS-UK 2023 meeting}, \hfill \textbf{23 Nov. 2023} \\
    Magdalen College, University of Oxford, UK
    %
    \item \textit{SPINS-UK 2022 meeting}, Jodrell Bank Observatory, UK
    \hfill \textbf{2 Nov. 2022}
    %
    \item \textit{Institute for Nuclear Theory Program 24-89a},
    \hfill \textbf{18 Jul. 2022} \\
    University of Washington, Seattle, USA
    %
    \item \textit{23rd International Conference on General Relativity and
    Gravitation}, \hfill \textbf{6 Jul. 2022} \\
    Chinese Academy of Sciences, CN (online)
    %
    \item \textit{PHAROS Conference 2022}, La Sapienza University, Rome, IT
    \hfill \textbf{18 May 2022}
    %
    \item \textit{GWPAW 2021}, Albert Einstein Institute, Hannover, DE (online)
    \hfill \textbf{17 Dec. 2021}
    %
    \item \textit{21st BritGrav Conference} (online)
    \hfill \textbf{15 Apr. 2021}
    %
    \item \textit{30th Texas Symposium on Relativistic Astrophysics}
    \hfill \textbf{17 Dec. 2019} \\
    University of Portsmouth, UK
    %
    \item Joint \textit{22nd International Conference on General Relativity and
    Gravitation} \hfill \textbf{9 Jul. 2019} \\
    and \textit{13th Edoardo Amaldi Conference on Gravitational Waves},
    Valencia, ES
    %
    \item \textit{SPINS-UK 2019 meeting}, University College London, UK
    \hfill \textbf{31 May 2019}
\end{etaremune}

%%%%%%%%%%%%%%%%%%%%%%%%%%%%%%%%%%%%%%%%%%%%

% Local Variables:
% mode: latex
% TeX-master: "FabianGittins.tex"
% End:


\section{Teaching Experience}

\textbf{Instructor,} University of Southampton, UK
\vspace*{0.05in}
\begin{itemize}
    \item[] MATH1007/1009, Mathematical Methods for Physical Scientists
\hfill \textbf{Feb.--May 2024}
\end{itemize}
%
\textbf{Guest Lecturer,} University of Southampton, UK
\vspace*{0.05in}
\begin{itemize}
    \item[] MATH3072, Advanced Fluid Dynamics
\hfill \textbf{Oct. 2022, Oct. 2023}
    \item[] MATH3006, Relativity, Black Holes and Cosmology
\hfill \textbf{Apr. 2022}
\end{itemize}
%
\textbf{Teaching Assistant,} University of Southampton, UK
\hfill \textbf{Oct. 2017--May 2021}
\begin{itemize}
    \item[] MATH1054/1055, Mathematics for Engineering and the Environment
    %
    \item[] MATH1057, Dynamics and Relativity
    %
    \item[] MATH1058, Operational Research I and Mathematical Computing
    %
    \item[] MATH2045, Vector Calculus and Complex Variable Theory
    %
    \item[] MATH3018, Numerical Methods
    %
    \item[] MATH3087, Maths and Your Future
\end{itemize}
%
\textbf{Teaching Assistant,} King Edward's School, Birmingham, UK
\hfill \textbf{Jan.--Apr. 2016}
\vspace*{0.05in}
\begin{itemize}
    \item[] Physics (11--16 yr)
\end{itemize}

\section{Mentoring and Supervision}

\textbf{PhD student mentoring}
\begin{itemize}
    \item[] Thibeau Wouters, Utrecht University, NL
    \hfill \textbf{Oct. 2024--Present}
    %
    \item[] Rahime Matur, University of Southampton, UK
    \hfill \textbf{Jan. 2023--Sep. 2024}
    %
    \item[] Rhys Counsell, University of Southampton, UK
    \hfill \textbf{Sep. 2021--Sep. 2024}
    %
    \item[] Shanshan Yin, University of Southampton, UK
    \hfill \textbf{Sep. 2021--Sep. 2024}
    %
    \item[] Thomas Celora, University of Southampton, UK
    \hfill \textbf{Sep. 2021--Sep. 2023}
    %
    \begin{itemize}
        \vspace{-0.05in}
        \item[] Now postdoc at Institute of Space Sciences, Barcelona, ES
    \end{itemize}
\end{itemize}
%
\textbf{Master's student supervision}
\begin{itemize}
    \item[] Tobie Walraven, Utrecht University, NL
    \hfill \textbf{Sep. 2025--Present}
\end{itemize}

\section{Professional Activities, Outreach and Service}

\textbf{Virgo Collaboration, Member} \hfill \textbf{Oct. 2024--Present} \\
%
\textbf{Cosmic Explorer Consortium, Member} \hfill \textbf{May 2024--Present} \\
%
\textbf{Einstein Telescope Collaboration, Member}
\hfill \textbf{Sep. 2023--Present}

\textbf{International Astronomical Union, Junior member}
\hfill \textbf{May 2023--Present} \\
%
\textbf{European Astronomical Society, Member}
\hfill \textbf{Nov. 2024--Present} \\
%
\textbf{Royal Astronomical Society, Elected fellow}
\hfill \textbf{Jul. 2021--Present} \\
%
\textbf{International Society on General Relativity and Gravitation,} \\
\textbf{Lifetime member} \hfill \textbf{May 2021--Present} \\
%
\textbf{Institute of Physics, Member} \hfill \textbf{Apr. 2021--Present}
\begin{itemize}
    \item[] Gravitational Physics Group, Committee member
    \hfill \textbf{Oct. 2021--Sep. 2025}
\end{itemize}

\textbf{Conference organiser}
\vspace*{0.05in}
\begin{itemize}
    \item[]
    \href{https://sites.google.com/view/spins-uk-2024}{SPINS-UK 2024 meeting},
    University of Southampton
    \hfill \textbf{10--12 Sep. 2024} \\
    \hspace*{1em} Local organising committee, $\sim 40$ participants
    %
    \item[]
    \href{https://plan.events.mpg.de/event/133}%
    {Continuous gravitational waves and neutron stars workshop}, \\
    Albert Einstein Institute, Hannover, DE \hfill \textbf{17--20 Jun. 2024}\\
    \hspace*{1em} Scientific organising committee, $\sim 50$ participants
    %
    \item[]
    \href{https://iop.eventsair.com/gpm2024/}%
    {Gravitational Physics Annual Meeting},
    Institute of Physics, UK \hfill \textbf{18 Jan. 2024} \\
    \hspace*{1em} Scientific organising committee, $\sim 50$ participants
    %
    \item[]
    \href{https://sites.google.com/view/britgrav23}{23rd BritGrav Conference},
    University of Southampton, UK \hfill \textbf{13--14 Apr. 2023} \\
    \hspace*{1em} Scientific and local organising committee,
    $\sim 100$ participants
\end{itemize}

\textbf{Seminar organiser}
\vspace*{0.05in}
\begin{itemize}
    \item[] Gravity Seminar, University of Southampton, UK
    \hfill \textbf{Oct. 2021--Sep. 2024}
    %
    \item[] Weekly Gravity Reading Group, University of Southampton, UK
    \hfill \textbf{Jan.--Jul. 2021}
\end{itemize}

\textbf{Journal referee}
\vspace*{0.05in} \\
\hspace*{1em}
Astronomy and Astrophysics,
Classical and Quantum Gravity,
Journal of Cosmology and Astroparticle Physics,
Journal of Physics G,
Monthly Notices of the Royal Astronomical Society,
Nature Astronomy,
Physical Review D,
Physical Review Letters,
The Astrophysical Journal

\textbf{Project referee}
\vspace*{0.05in}
\begin{itemize}
    \item[] Postdoctoral project, University of Namur, BE \hfill \textbf{2025}
    %
    \item[] Open Fellowship,
    Engineering and Physical Sciences Research Council, UK \hfill \textbf{2024}
\end{itemize}

\textbf{Outreach}
\vspace*{0.05in}
\begin{itemize}
    \item[] Southampton Science and Engineering Festival
    \hfill \textbf{7 May 2022, 18 Mar. 2023} \\
    \hspace*{1em} Organised neutron-star exhibit for general public and
    coordinated team of 10 volunteers
    %
    \item[] Mathematical Challenge \hfill \textbf{Mar.--Apr. 2020} \\
    \hspace*{1em} Marked over 200 pupil entries
    %
    \item[] Maths and Physics Workshop \hfill \textbf{8 Nov. 2017} \\
    \hspace*{1em} Demonstrated for $\sim 100$ secondary-school pupils
\end{itemize}

\textbf{Press (selected)}
\vspace*{0.05in}
\begin{itemize}
    \item[]
    \href{https://www.ntvn.nl/2025/10/sporen-van-quarkmaterie-zwaartekrachtgolven/}%
    {\textit{Sporen van quarkmaterie in zwaartekrachtgolven?}}
    \hfill \textbf{1 Oct. 2025} \\
    Nederlands Tijdschrift voor Natuurkunde
    %
    \item[]
    \href{https://www.newscientist.com/article/2343788-lightest-neutron-star-ever-found-could-contain-compressed-quarks/}%
    {\textit{Lightest neutron star ever found could contain compressed quarks}},
    New Scientist \hfill \textbf{24 Oct. 2022}
    %
    \item[]
    \href{https://www.livescience.com/millimeter-tall-neutron-star-mountains.html}%
    {\textit{Neutron star `mountains' may be blocking our view of mysterious gravitational}}
    \hfill \textbf{21 Jul. 2021} \\
    \href{https://www.livescience.com/millimeter-tall-neutron-star-mountains.html}%
    {\textit{waves}}, Live Science
    %
    \item[]
    \href{https://www.theregister.com/2021/07/21/mountain_neutron_stars/}%
    {\textit{Mountains on neutron stars are not even a millimetre tall due to extreme}}
    \hfill \textbf{21 Jul. 2021} \\
    \href{https://www.theregister.com/2021/07/21/mountain_neutron_stars/}%
    {\textit{gravity}}, The Register
    %
    \item[]
    \href{https://www.independent.co.uk/space/neutron-stars-mountains-ligo-virgo-b1886426.html}%
    {\textit{Scientists find tiny mountains on neutron stars that are a fraction of a}}
    \hfill \textbf{19 Jul. 2021} \\
    \href{https://www.independent.co.uk/space/neutron-stars-mountains-ligo-virgo-b1886426.html}%
    {\textit{millimetre tall}}, The Independent
    %
    \item[]
    \href{https://gizmodo.com/neutron-stars-have-mountains-that-are-less-than-a-milli-1847309049}%
    {\textit{Neutron Stars Have Mountains That Are Less Than a Millimeter Tall}},
    Gizmodo \hfill \textbf{18 Jul. 2021}
    %
    \item[]
    \href{https://www.newscientist.com/article/2278363-neutron-stars-are-remarkably-smooth-thanks-to-their-intense-gravity/}%
    {\textit{Neutron stars are remarkably smooth thanks to their intense gravity}},
    \hfill \textbf{24 May 2021} \\
    New Scientist
    %
    \item[]
    \href{https://astrobites.org/2018/11/16/ns_breakup/}%
    {\textit{Why don't they just break up?}}
    Astrobites \hfill \textbf{16 Nov. 2018}
\end{itemize}

\section{Computer Skills}

Advanced in Julia, Python.
Intermediate in Bash, C++, Mathematica, MATLAB.
Intermediate in high-performance computing (HTCondor, Slurm).
Markup languages: \LaTeX, Markdown.

\textbf{Software---}%
Most contributions can be found at \url{https://github.com/fgittins}.
Member of the \textit{Bilby} development team
(\url{https://github.com/bilby-dev/bilby}).
Contributor to \textit{SciML} (\url{https://sciml.ai}),
in particular \texttt{NonlinearSolve.jl}
(\url{https://github.com/SciML/NonlinearSolve.jl}).
Author of \texttt{RealisticSeismology} Julia code
(\url{https://github.com/fgittins/RealisticSeismology}).

\ifx\nopubs\undefined
\newcommand{\arxiv}[1]{[\href{http://arxiv.org/abs/#1}{arXiv:#1}]}
\def\zero{0}
\def\one{1}
\newcommand{\citeCount}[1]{%
    \def\val{#1}
    \ifx\val\zero%
    \else%
        \ifx\val\one%
        (1~citation)%
        \else%
        (#1~citations)%
        \fi%
    \fi
}

% Comment out to show, uncomment to hide
% \renewcommand{\citeCount}[1]{}

\def\apj{Astrophys.\ J.}
\def\apjs{Astrophys.\ J.\ Suppl.\ Ser.}
\def\cqg{Class.\ Quantum\ Gravity}
\def\joss{J.\ Open\ Source\ Softw.}
\def\mnras{Mon.\ Not.\ R.\ Astron.\ Soc.}
\def\prd{Phys.\ Rev.\ D}
\def\prl{Phys.\ Rev.\ Lett.}

\newcounter{numPubs}
\newcounter{pubCounter}

\setcounter{numPubs}{24}
\setcounter{pubCounter}{\value{numPubs}}

% \section{Publications in Progress}
% \secstartswithlist{}%
% \begin{etaremune}[start=\value{pubCounter}]
% \item
%   McNees,~R.
%   {\bf Stein,~L.~C.},
%   (2019)
%   {\it Cosmological perturbations in dynamical Chern-Simons}.
%   \setcounter{pubCounter}{\value{enumi}}
% \end{etaremune}

%%%%%%%%%%
%%%%%%%%%%
%%%%%%%%%%
%%%%%%%%%%
\newif\ifshowpubsummary
%%% Comment out the next line to omit the publication summary
\showpubsummarytrue
%%%
\ifshowpubsummary
\section{Publication Summary}

A full list of publications can be found on
\href{https://scholar.google.com/citations?hl=en&user=jET8KxgAAAAJ}%
{Google Scholar},
\href{https://inspirehep.net/authors/1798261}%
{INSPIRE-HEP}
and
\href{https://ui.adsabs.harvard.edu/user/libraries/SDMGXaKoRomaOJsefpo2yQ}%
{NASA ADS}.

\textbf{h-index---}%
As of 2025-10-03: 12 (according to Google Scholar),
11 (according to INSPIRE-HEP)
or 10 (according to NASA ADS).

\textbf{Top five cited---}%
Excluding long-author papers.
Citation counts from Google Scholar.
\begin{enumerate}
    \item \textbf{Gittins, F.}, Andersson, N., Jones, D.~I.,
    \textit{Modelling neutron star mountains},
    \href{https://doi.org/10.1093/mnras/staa3635}%
    {\mnras\ \textbf{500}, 5570 (2021)}
    \arxiv{2009.12794}.
    \citeCount{71}
    %
    \item \textbf{Gittins, F.}, Andersson, N.,
    \textit{Modelling neutron star mountains in relativity},
    \href{https://doi.org/10.1093/mnras/stab2048}%
    {\mnras\ \textbf{507}, 116 (2021)}
    \arxiv{2105.06493}.
    \citeCount{58}
    %
    \item \textbf{Gittins, F.}, Andersson, N., Pereira, J.~P.,
    \textit{Tidal deformations of neutron stars with elastic crusts},
    \href{https://doi.org/10.1103/PhysRevD.101.103025}%
    {\prd\ \textbf{101}, 103025 (2020)}
    \arxiv{2003.05449}.
    \citeCount{45}
    %
    \item \textbf{Gittins, F.}, Andersson, N.,
    \textit{Tidal deformations of hybrid stars with sharp phase transitions and
    elastic crusts},
    \href{https://doi.org/10.3847/1538-4357/ab8aca}%
    {\apj\ \textbf{895}, 28 (2020)}
    \arxiv{2003.10781}.
    \citeCount{37}
    %
    \item \textbf{Gittins, F.}, Andersson, N.,
    \textit{Population synthesis of accreting neutron stars emitting
    gravitational waves},
    \href{https://doi.org/10.1093/mnras/stz1719}%
    {\mnras\ \textbf{488}, 99 (2019)}
    \arxiv{1811.00550}.
    \citeCount{31}
\end{enumerate}
\else% don't show pub summary
\fi

\renewcommand{\citeCount}[1]{}

\section{Submitted Publications}

\secstartswithlist{}%
%\addtocounter{pubCounter}{-1}%
\begin{etaremune}[start=\value{pubCounter}]
    \item Yin, S., Andersson, N., \textbf{Gittins, F.},
    \textit{A post-Newtonian approach to neutron star oscillations}
    \arxiv{2504.06918}.
    \citeCount{2}
%
    \setcounter{pubCounter}{\value{enumi}}
\end{etaremune}

\section{Accepted Publications}

\secstartswithlist{}%
\addtocounter{pubCounter}{-1}%
\begin{etaremune}[start=\value{pubCounter}]
    \item Abac, A. \textit{et al.},
    \textit{The Science of the Einstein Telescope}
    \arxiv{2503.12263}.
    \citeCount{36}
%
    \setcounter{pubCounter}{\value{enumi}}
\end{etaremune}

% \section{Collaboration Publications}
% From 2008--2012, I was coauthor on 34 refereed LIGO and/or LIGO/Virgo
% collaboration publications. I only list short author-list publications below.

\section{Refereed Publications}

\secstartswithlist{}%
\addtocounter{pubCounter}{-1}%
\begin{etaremune}[start=\value{pubCounter}]
    \item Pnigouras, P., Andersson, N., \textbf{Gittins, F.}, Counsell, A.~R.,
    \textit{Dynamical neutron star tides: the signature of a mode resonance},
    \href{https://doi.org/10.1093/mnras/staf1285}%
    {\mnras\ \textbf{542}, 1375 (2025)}
    \arxiv{2508.06416}.
    \citeCount{1}
    %
    \item Counsell, A.~R., \textbf{Gittins, F.} \textit{et al.}
    \textit{Interface modes in inspiralling neutron stars: A gravitational-wave
    probe of first-order phase transitions},
    \href{https://doi.org/10.1103/8hvq-6dy7}%
    {\prl\ \textbf{135}, 081402 (2025)}
    \arxiv{2504.06181}.
    \citeCount{6}
    %
    \item \textbf{Gittins, F.}, Andersson, N., Yin, S.,
    \textit{Perturbation theory for post-Newtonian neutron stars},
    \href{https://doi.org/10.1088/1361-6382/ade83f}%
    {\cqg\ \textbf{42}, 135014 (2025)}
    \arxiv{2503.03345}.
    \citeCount{2}
    %
    \item \textbf{Gittins, F.}, Andersson, N.,
    \textit{Neutron-star seismology with realistic, finite-temperature nuclear
    matter},
    \href{https://doi.org/10.1103/PhysRevD.111.083024}%
    {\prd\ \textbf{111}, 083024 (2025)}
    \arxiv{2406.05177}.
    \citeCount{2}
    %
    \item \textbf{Gittins, F.} \textit{et al.},
    \textit{Problematic systematics in neutron-star merger simulations},
    \href{https://doi.org/10.1103/PhysRevD.111.023049}%
    {\prd\ \textbf{111}, 023049 (2025)}
    \arxiv{2409.13468}.
    \citeCount{5}
    %
    \item Counsell, A.~R., \textbf{Gittins, F.} \textit{et al.},
    \textit{Neutron star g modes in the relativistic Cowling approximation},
    \href{https://doi.org/10.1093/mnras/stae2721}%
    {\mnras\ \textbf{536}, 1967 (2025)}
    \arxiv{2409.20178}.
    \citeCount{6}
    %
    \item Counsell, A.~R., \textbf{Gittins, F.}, Andersson, N.,
    \textit{The impact of nuclear reactions on the neutron-star g-mode spectrum},
    \href{https://doi.org/10.1093/mnras/stae1242}%
    {\mnras\ \textbf{531}, 1721 (2024)}
    \arxiv{2310.13586}.
    \citeCount{11}
    %
    \item Pnigouras, P., \textbf{Gittins, F.}, \textit{et al.},
    \textit{The dynamical tides of spinning Newtonian stars},
    \href{https://doi.org/10.1093/mnras/stad3593}%
    {\mnras\ \textbf{527}, 8409 (2024)}
    \arxiv{2205.07577}.
    \citeCount{16}
    %
    \item Beri, A. \textit{et al.},
    \textit{AstroSat and NuSTAR observations of XTE J1739-285 during the
    2019-2020 outburst},
    \href{https://doi.org/10.1093/mnras/stad902}%
    {\mnras\ \textbf{521}, 5904 (2023)}
    \arxiv{2303.13085}.
    \citeCount{10}
    %
    \item \textbf{Gittins, F.} \textit{et al.},
    \textit{Modelling Neutron-Star Ocean Dynamics},
    \href{https://doi.org/10.3390/universe9050226}%
    {Universe\ \textbf{9}, 226 (2023)}
    \arxiv{2304.05413}.
    \citeCount{4}
    %
    \item \textbf{Gittins, F.}, Andersson, N.,
    \textit{The r-modes of slowly rotating, stratified neutron stars},
    \href{https://doi.org/10.1093/mnras/stad672}%
    {\mnras\ \textbf{521}, 3043 (2023)}
    \arxiv{2212.04892}.
    \citeCount{16}
    %
    \item Andersson, N., \textbf{Gittins, F.},
    \textit{Formulating the r-mode Problem for Slowly Rotating Neutron Stars},
    \href{https://doi.org/10.3847/1538-4357/acbc1e}%
    {\apj\ \textbf{945}, 139 (2023)}
    \arxiv{2212.04837}.
    \citeCount{13}
    %
    \item Andersson, N., \textbf{Gittins, F.} \textit{et al.},
    \textit{Building post-Newtonian neutron stars},
    \href{https://doi.org/10.1088/1361-6382/acace5}%
    {\cqg\ \textbf{40}, 025016 (2023)}
    \arxiv{2209.05871}.
    \citeCount{2}
    %
    \item Riley, J. \textit{et al.},
    \textit{Rapid Stellar and Binary Population Synthesis with COMPAS},
    \href{https://doi.org/10.3847/1538-4365/ac416c}%
    {\apjs\ \textbf{258}, 34 (2022)}
    \arxiv{2109.10352}.
    \citeCount{2}
    %
    \item \textbf{Gittins, F.}, Andersson, N.,
    \textit{Modelling neutron star mountains in relativity},
    \href{https://doi.org/10.1093/mnras/stab2048}%
    {\mnras\ \textbf{507}, 116 (2021)}
    \arxiv{2105.06493}.
    \citeCount{58}
    %
    \item \textbf{Gittins, F.}, Andersson, N., Jones, D.~I.,
    \textit{Modelling neutron star mountains},
    \href{https://doi.org/10.1093/mnras/staa3635}%
    {\mnras\ \textbf{500}, 5570 (2021)}
    \arxiv{2009.12794}.
    \citeCount{71}
    %
    \item \textbf{Gittins, F.}, Andersson, N., Pereira, J.~P.,
    \textit{Tidal deformations of neutron stars with elastic crusts},
    \href{https://doi.org/10.1103/PhysRevD.101.103025}%
    {\prd\ \textbf{101}, 103025 (2020)}
    \arxiv{2003.05449}.
    \citeCount{45}
    %
    \item \textbf{Gittins, F.}, Andersson, N.,
    \textit{Tidal deformations of hybrid stars with sharp phase transitions and
    elastic crusts},
    \href{https://doi.org/10.3847/1538-4357/ab8aca}%
    {\apj\ \textbf{895}, 28 (2020)}
    \arxiv{2003.10781}.
    \citeCount{37}
    %
    \item \textbf{Gittins, F.}, Andersson, N.,
    \textit{Population synthesis of accreting neutron stars emitting
    gravitational waves},
    \href{https://doi.org/10.1093/mnras/stz1719}%
    {\mnras\ \textbf{488}, 99 (2019)}
    \arxiv{1811.00550}.
    \citeCount{31}
%
    \setcounter{pubCounter}{\value{enumi}}
\end{etaremune}

\section{Review Articles}

\secstartswithlist{}%
\addtocounter{pubCounter}{-1}%
\begin{etaremune}[start=\value{pubCounter}]
    \item \textbf{Gittins, F.},
    \textit{Gravitational waves from neutron-star mountains},
    \href{https://doi.org/10.1088/1361-6382/ad1c35}%
    {\cqg\ \textbf{41}, 043001 (2024)}
    \arxiv{2401.01670}.
    \citeCount{16}
%
    \setcounter{pubCounter}{\value{enumi}}
\end{etaremune}

\section{Software Articles}

\secstartswithlist{}%
\addtocounter{pubCounter}{-1}%
\begin{etaremune}[start=\value{pubCounter}]
    \item Riley, J. \textit{et al.},
    \textit{COMPAS: A rapid binary population synthesis suite},
    \href{https://doi.org/10.21105/joss.03838}%
    {\joss\ \textbf{7}, 3838 (2022)}.
    \citeCount{18}
%
    \setcounter{pubCounter}{\value{enumi}}
\end{etaremune}

\section{Conference Proceedings}

\secstartswithlist{}%
\addtocounter{pubCounter}{-1}%
\begin{etaremune}[start=\value{pubCounter}]
    \item Thomas, A. Stevenson, E., \textbf{Gittins, F.} \textit{et al.},
    \textit{Galactic Archaeology with TESS: Prospects for Testing the Star
    Formation History in the Solar Neighbourhood},
    \href{https://doi.org/10.1051/epjconf/201716005006}%
    {EPJ\ Web\ Conf.\ \textbf{160}, 05006 (2017)}
    \arxiv{1610.08862}.
    \citeCount{1}
\end{etaremune}

%%%%%%%%%%%%%%%%%%%%%%%%%%%%%%%%%%%%%%%%%%%%

% Local Variables:
% mode: latex
% TeX-master: "FabianGittins.tex"
% End:

\else
%
\fi

\section{References}

\vspace*{0.05in}
\parbox{\textwidth}{%
\textbf{Prof Nils Andersson,} Professor of Applied Mathematics \\
School of Mathematical Sciences \\
University of Southampton, \\
University Road \\
Southampton, SO17 1BJ \\
United Kingdom \\
email: \href{mailto:n.a.andersson@soton.ac.uk}{n.a.andersson@soton.ac.uk} \\
office phone: \href{tel:+44 23 8059 4551}{+44 23 8059 4551}}
\par
%
\parbox{\textwidth}{%
\textbf{Prof Chris van den Broeck,} Professor of Physics \\
Institute of Gravitational and Subatomic Physics \\
Utrecht University \\
Princetonplein 1 \\
3584 CC Utrecht \\
The Netherlands \\
email: \href{mailto:c.f.f.vandenbroeck@uu.nl}{c.f.f.vandenbroeck@uu.nl} \\
office phone: \href{tel:+31 6 25 13 3968}{+31 6 25 13 3968}}
\par
%
\parbox{\textwidth}{%
\textbf{Dr David Tsang,} Lecturer in Physics \\
Department of Physics \\
University of Bath \\
Claverton Down \\
Bath, BA2 7AY \\
United Kingdom \\
email: \href{mailto:d.tsang@bath.ac.uk}{d.tsang@bath.ac.uk}\\
office phone: \href{tel:+44 12 2538 4539}{+44 12 2538 4539}}

\end{resume}
\end{document}

% Local Variables:
% mode: latex
% End:
